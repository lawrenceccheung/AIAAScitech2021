\documentclass[conf]{new-aiaa}
%\documentclass[journal]{new-aiaa} for journal papers
\usepackage[utf8]{inputenc}

\usepackage{graphicx}
\usepackage{amsmath}
\usepackage[version=4]{mhchem}
\usepackage{siunitx}
\usepackage{longtable,tabularx}
\usepackage{fixme}
\setlength\LTleft{0pt}
\hypersetup{
  colorlinks=true
}
\usepackage{tikz}

\newcommand{\average}[1]{\langle #1 \rangle}

% Setup fixme env for collaborative notes
\fxsetup{status=draft, layout=inline, theme=color}

\title{Computation and comparison of the stable northeastern United States
  marine boundary layer}

\author{Lawrence C. Cheung\footnote{Principal member of technical staff, Thermal/Fluid Science \& Engineering, AIAA member}}
\affil{Sandia National Laboratories, Livermore, CA 94550}

\author{
  Shreyas Ananthan\footnote{Researcher V, High-Performance Algorithms and Complex Fluids, AIAA Member},
  Michael J. Brazell\footnote{Researcher III, High-Performance Algorithms and Complex Fluids, AIAA Member},
  Ganesh Vijayakumar\footnote{Researcher III, Mechanical Engineering, AIAA Member}, and
}
\affil{National Renewable Energy Laboratory, Golden CO 80401}

\author{Nathaniel B. deVelder\footnote{Postdoctoral appointee, Wind
    Energy Technologies Department, AIAA member} and Alan
  S. Hsieh.\footnote{Senior member of technical staff, Wind Energy
    Technologies Department, AIAA member}} \affil{Sandia National
  Laboratories, Albuquerque, NM 87185}

\begin{document}

\maketitle

\begin{abstract}
  In this work we investigate the behavior of stable marine boundary layers
  located near the coast of the Northeastern United States.  Using the ExaWind
  large eddy simulation (LES) codes, three stable atmospheric conditions were
  chosen to match the Cape Wind measurements of Archer \emph{et al.} with wind
  speeds of 5 m/s, 10 m/s, and 15 m/s at the 20 m measurement height.  The
  behavior of the stable boundary layers, including mean flow quantities and
  turbulent statistics, are examined and compared to previous computations of
  the neutral and unstable offshore boundary layer at the same location.  This
  study also examines the domain and mesh requirements necessary to capture the
  turbulent scales for the stable offshore boundary layers.  Finally, we compare
  solutions computed using both Nalu-Wind and AMR-Wind solvers, and
  compare their predicted solutions and performance in this study.

\end{abstract}

\section{Nomenclature}
{\renewcommand\arraystretch{1.0}
  \noindent\begin{longtable*}{@{}l @{\quad=\quad} l@{}}
ABL       & Atmospheric Boundary Layer \\
$\alpha$  & Wind shear exponent  \\
$\beta$   & Thermal expansion coefficient \\
$c_p$     & Specific heat capacity \\
$C_\epsilon$  & Turbulence model constant \\
$\Delta, \Delta_{x,y,z}$  & Grid spacing, grid spacing in x, y, z directions \\
$\epsilon_{ijk}$ & Levi-Civita symbol\\
$f$       & Frequency \\
$f_{max}$  & Maximum resolvable frequency due to mesh limits \\
$f_{Ny}$   & Nyquist frequency \\
$g, g_i$   & Gravitational acceleration \\
$h$       & Enthalpy \\
$k, TKE$  & Turbulent kinetic energy \\
$L$       & Turbulent integral length scale \\
$L_{Ob}$   & Obukhov length scale \\
$\nu_t$   & Turbulent viscosity \\
LES       & Large Eddy Simulation \\
$n_{dofs}$ & Number of degrees of freedom \\
$n_r$     & Number MPI ranks \\
$\phi_h$  & Potential temperature gradient \\
$q_j$     & Energy flux vector\\
$\rho$    & Density \\
$\rho_\circ$ & Reference density  \\
$Pr_t$    & Turbulent Prandtl number \\
$R$       & Universal gas constant \\
$R_{ij}$   & Two point correlation tensor \\
SBL       & Stable Boundary Layer \\
$S^u_i$    & ABL forcing source term \\
$S_i$     & Wind spectra \\
$\sigma_i$ & Wind speed variance \\
$\tau_{ij}$ & Stress tensor \\
$\theta$  & Potential temperature \\
$T$       & Absolute temperature \\
$TI$      & Turbulence intensity \\
$t_{ts}$  & Time per time-step \\
$\overline{U}_{horiz}$ & Time averaged horizontal velocity \\
$u_i$     & Wind velocity in the $i$ direction  \\
$u_\tau$   & Friction velocity    \\
$V_1$     & Resolved wind speed at first grid level \\
$\Omega_i$ & Earth angular velocity vector
\end{longtable*}}

\section{Introduction}

\lettrine{T}he planned installation of several offshore wind energy
plants in the United States has highlighted the need to properly
understand the wind resource and atmospheric characteristics of the U.S.
Atlantic Coast.  Several atmospheric phenomena particular to this
region, such as coastal low-level jets or seasonal Nor’easters, have
the potential to substantially impact the operation and power
production of offshore wind plants.  Of particular interest to the
current study is the atmospheric stability of the marine boundary
layer in the Northeastern United States.

Atmospheric stability plays a large role in determining the power
production of wind plants because it directly affects the vertical
distribution of momentum and turbulent energy in the atmospheric
boundary layer (ABL).  The differences between the stable, neutral, or
unstable stratified ABL can lead to large changes in wind speed or
turbulence profiles, and ultimately change the operation of wind
turbines.  Atmospheric stability may also play a role in the formation
of low-level jets \cite{nunalee2014mesoscale} and cause increased
fatigue loads on offshore wind turbines.

Several recent measurement campaigns have provided data to understand
the ABL and wind characteristics for potential offshore wind farms in
the Northeastern United States.  Pichugina et al. \cite{pichugina2017properties}
measured the wind profiles and vertical shear profiles in the Gulf of
Maine using a ship-borne LIDAR approach.  Analysis of measured data
sets from Nantucket Sound by Archer et
al. \cite{archer2016predominance} showed a predominance of low-shear,
unstable conditions at the site.  However, strong seasonal variations
and stratification changes due to diurnal variation were also
observed.

\subsection{Previous Simulation Work}

In addition to these measurement campaigns, large eddy simulations (LES) have
also been used to study ABL stability characteristics.  Recent work by
Kaul et al. \cite{kaul2020large} has shown that LES computations using
Nalu-Wind can capture the neutral and convectively unstable onshore
ABL.  Previous simulations by Cheung et al. \cite{cheung2020large}
successfully replicated the unstable and neutral ABL corresponding to
the Cape Wind meteorological tower measurements
\cite{archer2016predominance} and showed the effects of surface
heating on atmospheric stratification. LES of the stable ABL (SBL) presents
certain challenges compared to neutral and convective stability states, stemming
primarily from small dominant eddy length scales and low turbulence intensity.
These challenges require higher resolution grids and/or advanced subgrid
scale (SGS) models. As such, LES research of the SBL has been slower
to evolve than the other stability states.

The first successful LES of the SBL was published by Mason and Derbyshire in
1990 \cite{Mason1990}. They used a strategy of simulating a near steady-state
neutral stability turbulent simulation, then added a cooling-surface boundary
to develop the stable ABL. They recorded a need for implausibly strong initial
perturbations to sustain turbulence using the Smagorinsky SGS model. Brown et
al.\ \cite{Brown1994} worked with Mason and Derbyshire to add stochastic backscatter
to the SGS model, resulting in better agreement with surface layer observations.
Andren \cite{Andren1995} simulated two SBL types, one with a neutral layer aloft, and
one capped by an inversion, comparing two SGS models. He found significant improvements
with the Sullivan SGS model, on par with the earlier backscatter model. Kemp and
Thompson \cite{Kemp1996} used the same model as \cite{Brown1994} to study particle
dispersion, but the LES code did not replicate an expected plume growth. Derbyshire
wrote a review of SBL modeling along with future projections in 1999 \cite{Derbyshire1999},
noting the importance of local Richardson number and identifying a need to go
to higher resolutions in SBL research.

In 2000, Kosovic and Curry \cite{Kosovic2000} used an LES simulation
with a nonlinear SGS model containing backscatter to favorably compare
to the Beaufort Sea Arctic Stratus Experiment (BASE) measurements,
showing that the SBL could reach a quasi steady-state. They also
showed good agreement with the Nieuwstadt analytical model. In the same
year, Saiki et al. \cite{Saiki2000} simulated an SBL with
a strong capping inversion at $z$=500 m, producing reasonable surface layer profiles and
identifying a need for SBL specific SGS models. Beare and MacVean \cite{Beare2004}
presented the first grid-focused work on LES of the SBL, using grids with deltas
between 2 m and 12.5 m, finding significant variation between grid sizes. The inclusion
of a backscatter model resulted in significant variation at a 6.25 m resolution, but
very little at a 2 m resolution. Basu et al. \cite{Basu2005} published a set of
turbulence statistics, measurements, and LES simulations in support
of the viability of Nieuwstadt's local scaling hypothesis under very stable
conditions. Jimenez and Cuxart \cite{Jimenez2005} set out to quantify the range
for which standard SGS models, using the Kolmogorov theory for dissipation, were
valid for increasingly stable flows. They found the model to perform adequately
for weakly stable, and some moderately stable flow regimes.

An inter-comparison of LES simulations of the SBL was organized in 2006, the
results of which are published in Beare et al.\ \cite{Beare2006}. Collectively,
this is known as the GABLS initiative, and involved simulating a moderately
stable flow taken from Kosovic and Curry \cite{Kosovic2000}. Eleven codes
were compared in all, wherein grid deltas of less than 12.5 m were found to
successfully maintain resolved turbulence, grid deltas less than 3.125 m
showed little statistical change, and results became independent of SGS
modeling at 1 m resolution. In the same year, Kumar et al. \cite{Kumar2006}
successfully simulated an entire diurnal cycle with LES using a Lagrangian
dynamic scale dependent SGS model. This model has been used by several
researchers since 2011, including Lu and Porte-Agel \cite{Lu2011}, Huang
and Bou-Zeid \cite{Huang2013}, Park et al.\ \cite{Park2014}, and
Abkar and Porte-Agel \cite{Abkar2015}. Stoll and Porte-Agel \cite{Stoll2007}
compared three grid refinements and three scale dependent dynamic SGS models,
showing that the Lagrangian and locally averaged models reproduced important
flow features. The work of van Dop and Axelsen \cite{vanDop2007} provided
even more evidence in support Nieuwstadt's local scaling hypothesis for SBLs.
Chow and Zhou \cite{Chow2011} showed evidence that the Smagorinsky model is
simply too dissipative for use in strongly stable flows.

More recent work in LES of the SBL includes Matheou \cite{Matheou2016} who
explored a large parameter space using the Smagorinsky-Lilly SGS
closure, as well as Sullivan et al.\ \cite{sullivan2016turbulent}. Several researchers
have applied SBL simulation in the wind energy space \cite{Aitken2014, Abkar2015, Ghaisas2017}.
In 2020, Wurps et al. \cite{Wurps2020} provided a rigorous grid resolution study for the SBL,
looking at mean flow, turbulence variables, and two-point correlations, using a dynamic
SGS model that allows backscatter. They found grid requirements to be 2.5 m, 10 m, and 20 m
for stable, neutral, and convective boundary layers respectively. They advocate looking
at two point correlations in order to quantify flow structure size, which in turn
provides a basis for evaluating grid resolution. In all of the reviewed literature, SBL LES
has been run in the onshore context only; a complete comparison
including offshore stable ABL conditions has yet to be completed.

\subsection{Objectives}
In the current study on stable marine boundary layers, we aim to build
on the previous work in the literature and complete several
objectives.  First, we examine the domain and mesh requirements
necessary to capture the turbulent scales for the stable offshore
boundary layers.  Secondly, two incompressible-flow LES solvers within
the ExaWind simulation suite will be compared in this study: The
unstructured-grid solver Nalu-Wind and the structured, adaptive mesh
refinement (AMR) solver AMR-Wind.  Both codes are used to calculate
the same SBL, and the relative differences between the codes are
reported.  Finally, a series of stable boundary layer conditions will
be computed using AMR-Wind that match the measurements of Archer et
al. \cite{archer2016predominance}.  The behavior of the stable
boundary layers, including mean flow quantities and turbulent
statistics, are examined and compared to the neutral and unstable
cases computed in \cite{cheung2020large}.

%%%%%%%%%%%%%%%%%%%%%%%%%%
%% SECTION 2: METHODOLOGY
%%%%%%%%%%%%%%%%%%%%%%%%%%
\section{Methodology}
Write something about what we're comparing with, the codes we're
using, and how we set up the simulations.

\subsection{Measured offshore conditions}
% \fxnote{Lawrence write this part}

Following the work of Cheung et al \cite{cheung2020large}, we use
measurements of the offshore coastal marine boundary layer, collected
by the Cape Wind meteorological tower in Nantucket Sound, as the basis
for this computational study.  The Cape Wind platform collected wind
measurements at 20m, 41m, and 60m above the mean water level, along
with temperature and barometric measurements from the years 2003-2011.
From the observations, Archer et al.\cite{archer2016predominance}
found that the marine boundary layer is predominantly unstable, with
61\% of conditions classified as unstable, versus 21\% neutral and
18\% stable.  The stratification of the marine boundary layer, as
determined by the Obukhov length, also had a large impact on the wind
speed profile, with flatter, non-logarithmic profiles seen during
unstable conditions.

From the measured distribution of atmospheric stabilities, turbulent
kinetic energies (TKE), and wind speeds from the Cape Wind platform at
z=20m, a stable set of conditions were chosen as targets for this
computational study.  A summary of the conditions for all stability
classes is given table \ref{tab:CapeWindMeasurements}. In this work we
focus on the three stable atmospheric conditions at wind speeds of
5m/s, 10m/s, and 15m/s, while the neutral and unstable conditions
where studied in previous work \cite{cheung2020large}.

To maintain consistency with the measured data, the turbulence
intensity (TI) is calculated using the TKE as
\begin{equation}
  \textrm{TI} =
  \frac{\sqrt{\frac{2}{3}\times\textrm{TKE}}}{\overline{U}_{horiz}} =
  \frac{\sqrt{\frac{1}{3}\left( \overline{u'u' + v'v' + w'w'}
      \right)}}{\overline{U}_{horiz}}
\end{equation}
The averaged wind speeds were enforced at the measurement height of
20m, and the applied wind directions were consistent with the
predominant stable direction of 225 degrees southwest.

\begin{table}
\caption{\label{tab:CapeWindMeasurements} Measured conditions at Cape
  Wind.  The stable atmospheric conditions used in this study are
  highlighted in bold below.} \centering
\begin{tabular}{cccc}
  \hline
  Stability    & Wind speed [m/s] & Wind dir [deg] & Turbulence intensity \\
  \hline
  Neutral      & 5                & 225            & 0.055           \\
  Neutral      & 10               & 225            & 0.055           \\
  Neutral      & 15               & 225            & 0.065           \\
  Unstable     & 5                & 315            & 0.080           \\
  Unstable     & 10               & 315            & 0.075           \\
  Unstable     & 15               & 315            & 0.090           \\
  \bf{Stable}  & \bf{5}           & \bf{225}       & \bf{0.045}      \\
  \bf{Stable}  & \bf{10}          & \bf{225}       & \bf{0.050}      \\
  \bf{Stable}  & \bf{15}          & \bf{225}       & \bf{0.060}      \\
\hline
\end{tabular}
\end{table}


\subsection{Computational methodology}
Provide a general description of the LES codes that we use.

\subsection{LES formulation}

\subsubsection{Governing equations}

\subsubsection{\label{sec:wallmodelBC}Lower wall model BC}
\fxnote{Ganesh write this part}\\
What we implemented in terms of the lower boundary condition.

\subsubsection{Nalu-Wind}
\fxnote{Shreyas write this part}\\
Describe Nalu-Wind

\subsubsection{AMR-Wind}
\fxnote{Mike write this part}\\
Describe AMR-Wind

\subsection{Computational setup}
% \fxnote{Domains, grids, boundary conditions}


\begin{table}
\caption{\label{tab:z0tempparam} LES parameters for stable ABL conditions}
\centering
\begin{tabular}{ccccc}
  \hline
  Stability & Wind speed & Surface roughness $z_0$ & Surface
  temperature change & timestep\\
  \hline
  Stable       & 5  m/s           & 0.0005 m       & -0.32 K/hr   & 0.25 sec   \\
  Stable       & 10 m/s           & 0.0005 m       & -1.40 K/hr   & 0.125 sec  \\
  Stable       & 15 m/s           & 0.0005 m       & -1.50 K/hr   & 0.0625 sec \\
\hline
\end{tabular}
\end{table}

The computational methodology for the AMR-Wind and Nalu-Wind LES codes
follows practices similar to the previous offshore ABL study, with
some modifications to handle the stable stratification.  As in
\cite{cheung2020large}, the domain was a square prism geometry, with
the $x$ and $y$ coordinates aligned in the East and North directions,
respectively.  The domain size and mesh requirements were determined
through a numerical grid study.  As discussed in section
\ref{sec:gridstudy}, the horizontal dimensions of 750m $\times$ 750m
were found to be sufficient to capture any large scale structures in
the boundary layer, and in all cases the vertical dimension of 1000m
was also used.  The grid study in section \ref{sec:gridstudy} also
showed that the stable stratification required greater mesh refinement
to capture the smaller turbulent scales, and a uniform cell size of
2.5m was adopted for the offshore LES computations across both codes.

Momentum source terms were included in Nalu-Wind and AMR-Wind to
ensure that the horizontally averaged velocity matched the targeted
wind speed at the z=20 m height.  These source terms are based on the
difference between the desired wind velocity and the instantaneous
horizontally averaged velocity, and are only a function of time and
height z.  Coriolis forcing matching the Cape Wind latitude was
included to capture the effect of wind change with elevation.

\subsubsection{Boundary and initial conditions}
In both horizontal directions of the computational domain, periodic
boundary conditions were applied.  At the lower boundary, we chose to
represent the air/ocean interface using flat boundary with small
amount of surface roughness.  This allowed the wall model discussed in
section \ref{sec:wallmodelBC} to be applied.  Monin-Obukhov similarity
theory was used to determine the velocity and temperature profiles
near the lower surface given a surface roughness height $z_0$ and the
prescribed surface temperature as a function of time.  At the upper
surface of the domain, a potential flow based boundary condition is
applied along with a normal temperature gradient of 0.003 K/m.

The initial temperature profile in all cases was a constant 300K until
the specified inversion height of 650m.  The inversion layer thickness
was 100m, and above this, the temperature linearly increased until it
reached 308.75K at the upper boundary.  The initial mean velocity
profile was uniform throughout the domain with superimposed sinusoidal
velocity perturbations of magnitude 1 m/s to promote the development
of turbulence.

\subsubsection{Determination of surface roughness and prescribed temperature }
To match the measured TI conditions given in table
\ref{tab:CapeWindMeasurements}, the surface roughness and prescribed
surface temperature change were adjusted through an initial
trial-and-error process.  The final surface roughness of $z_0$=0.0005m
used the stable offshore conditions (see table \ref{tab:z0tempparam})
matched the roughness used in the neutral 15m/s and unstable 5m/s and
10m/s cases from the previous study \cite{cheung2020large}.  This
value of surface roughness is consistent with the measurements from
the North Sea \cite{taylor2001dependence}, which found values of $z_0$
ranging from $5 \times 10^{-5}$m to $5\times 10^{-3}$m.

The values of the prescribed temperature change at the ocean surface
is also given in table \ref{tab:CapeWindMeasurements}.  As the wind
speed increases, a larger temperature decrease was used to maintain
similar levels of stable stratification, as expected from
Monin-Obukhov similarity theory.


%%%%%%%%%%%%%%%%%%%%%%%%%%
%% SECTION 3: RESULTS
%%%%%%%%%%%%%%%%%%%%%%%%%%
\section{Results}
\fxnote{This should be a collaborative section}

\subsection{Domain and grid resolution study}

%%%%%%%%%%%%%%% GRID STUDY: INTEGRAL LENGTH %%%%%%%%%%%%%%%%%%%%%%%%%%%%%%%%%%%
\begin{table}
\caption{\label{tab:GridStudySetup} The setup for the grid resolution study} \centering
\begin{tabular}{ccccc}
  \hline
  Case              & dx [m] & dy [m] & dz [m] \\
  \hline
  Nalu-wind coarse  &  10.0  & 10.0   & 2.5   \\
  Nalu-wind medium  &   5.0  &  5.0   & 2.5   \\
  Nalu-wind fine    &   2.5  &  2.5   & 2.5   \\
  AMR-wind fine     &   2.5  &  2.5   & 2.5   \\
\hline
\end{tabular}
\end{table}

The maximum resolvable frequency on a mesh with grid size $\Delta$ and
average horizontal wind speed $\bar{U}_{horiz}$ can be estimated as
\begin{equation}
  \label{eq:fmax}
  f_{max} = \frac{0.6\bar{U}_{horiz}}{N\Delta}.
\end{equation}
Here, equation \ref{eq:fmax} assumes that the turbulent eddies convect
with a velocity of $0.6\bar{U}_{horiz}$ and $N=8$ is chosen for the
purposes of this study.  Due to the alignment of the flow with the
mesh directions, the grid size $\Delta = \sqrt{2}\cdot \textrm{dx}$.

The wind spectra
\begin{equation}
  \int_0^\infty S_i(f) \textrm{d}f = \sigma_i^2
\end{equation}

%%%%%%%%%%% Grid resolution spectra figure %%%%%%%%%%%%%%%%%%%%%%%%%
% Created in Postprocessing/GridStudy/GridStudy_Spectra.ipynb
\begin{figure}[hbt!]
  \label{fig:GridStudySpectra}
  \centering
  \includegraphics[width=2.0in]{figures/GridStudy_Spectra_Su.png}
  \includegraphics[width=2.0in]{figures/GridStudy_Spectra_Sv.png}
  \includegraphics[width=2.0in]{figures/GridStudy_Spectra_Sw.png}
  \caption{Calculation of the wind spectra $S_i$ for LES of stable
    5m/s case with different resolutions.  The gray vertical lines
    correspond to the maximum resolvable frequency according to
    equation (\ref{eq:fmax}). }
\end{figure}
%%%%%%%%%%%%%%%%%%%%%%%%%%%%%%%%%%%%%%%%%%%%%%%%%%%%%%%%%%%%%%%%%%%%

The two-point correlation tensor defined as
\begin{equation}
  \label{eq:Rij}
  R_{ij}({\mathbf x},\boldsymbol{\xi}) = 
  \frac{\overline{ {u'_i(\mathbf{x}, t) u'_j(\mathbf{x}+\boldsymbol{\xi},t)} }}
       { \sqrt{\overline{ u'^2_i }} \sqrt{\overline{ u'^2_j}} }
\end{equation}
where the velocity fluctuations $u'_i$ are defined as 
\begin{equation}
  u'_i(\mathbf{x},t) = u_i(\mathbf{x},t) - \overline{ u_i(\mathbf{x},t) }
\end{equation}

The horizontally averaged correlation coefficient $\langle
R_{ij}(\boldsymbol{\xi})\rangle$ is computed from
$R_{ij}(\mathbf{x},\boldsymbol{\xi})$ by sampling over ...

%%%%%%%%%%% Grid resolution correlation figure %%%%%%%%%%%%%%%%%%%%
% Created in Postprocessing/GridStudy/GridStudy_Rij.ipynb
\begin{figure}[hbt!]
  \label{fig:GridStudyRij}
  \centering
  \includegraphics[width=3in]{figures/GridStudy_Rij_Longitudinal.png}
  \includegraphics[width=3in]{figures/GridStudy_Rij_Lateral.png}
  \caption{Calculation of the averaged longitudinal and lateral
    $R_{ij}(\xi)$ coefficient at $z$=20m for LES of stable 5m/s case
    with different resolutions.}
\end{figure}
%%%%%%%%%%%%%%%%%%%%%%%%%%%%%%%%%%%%%%%%%%%%%%%%%%%%%%%%%%%%%%%%%%%%


The turbulent integral lengthscale can be calculated from $\langle
R_{ij}(\xi) \rangle$ via
\begin{equation}
  L = \int_0^\infty \langle R_{ij}(\xi) \rangle \: {\textrm d}\xi
\end{equation}

%%%%%%%%%%%%%%% GRID STUDY: INTEGRAL LENGTH %%%%%%%%%%%%%%%%%%%%%%%%%%%%%%%%%%%
\begin{table}
\caption{\label{tab:GridStudyLscale} The calculated turbulent integral
  lengthscale for each of the grid resolutions} \centering
\begin{tabular}{ccccc}
  \hline
  Case              & dx [m] & Longitudinal L [m] & Lateral L [m] \\
  \hline
  Nalu-wind coarse  &  10.0  & 36.433999          & 0.0        \\
  Nalu-wind medium  &   5.0  & 21.485711          & 4.519247   \\
  AMR-wind fine     &   2.5  & 27.689340          & 9.332672   \\
\hline
\end{tabular}
\end{table}

\subsection{Comparison of AMR-Wind vs Nalu-Wind}

\subsection{ABL Physics}

\subsubsection{ABL integrated quantities}
Describe integrated quantities

\subsubsection{Horizontally averaged profiles}

\subsubsection{Wind spectra and turbulence statistics}

The Kaimal model for spectra \cite{kaimal1973turbulence, cheynet2017spectral}
\begin{equation}
  \label{eq:kaimal}
  \frac{fS_i}{u_\tau^2} = \frac{a_i(fz/\bar{U})}{\left(1+b_i(fz/\bar{U})^{\alpha_i}\right)^{\beta_i}}
\end{equation}

%%%%%%%%%%%%%%% KAIMAL MODEL PARAMETERS %%%%%%%%%%%%%%%%%%%%%%%%%%%%%%%%%%%
\begin{table}
\caption{\label{tab:KaimalParameters} Parameters for Kaimal model}
\centering
\begin{tabular}{ccccc}
  \hline
  Velocity & $a_i$ & $b_i$ & $\alpha_i$  & $\beta_i$ \\
  \hline
  $u$      & 105.0 & 33.0  & 1           & 5/3  \\
  $v$      &  17.0 &  9.5  & 1           & 5/3  \\
  $w$      &   2.1 &  5.3  & 5/3         &   1  \\
\hline
\end{tabular}
\end{table}

\subsubsection{Comparison with neutral and unstable conditions}


%%%%%%%%%%% All stability correlation figure %%%%%%%%%%%%%%%%%%%%
% created in Postprocessing/ABLLength/CompareAll_ABL_Lengthscales.ipynb 
\begin{figure}[hbt!]
  \label{fig:AllStabilityRij}
  \centering
  \includegraphics[width=3in]{figures/AllStability_Rij_Longitudinal.png}
  \includegraphics[width=3in]{figures/AllStability_Rij_Lateral.png}
  \caption{Calculation of the averaged longitudinal and lateral
    $R_{ij}(\xi)$ coefficient at $z$=20m for all ABL stability cases.}
\end{figure}
%%%%%%%%%%%%%%%%%%%%%%%%%%%%%%%%%%%%%%%%%%%%%%%%%%%%%%%%%%%%%%%%%%%%


\section{Conclusion}
In this study, we have simulated the stable offshore boundary layer matching the
characteristics of measurements from Nantucket Sound in the Northeastern United
States. Three cases with wind speeds of 5 m/s, 10 m/s, and 15 m/s were
selected, and simulations were performed using the ExaWind codes to match the TI
and stability levels observed at the Cape Wind platform.  When examining
the effect of stratification using the Obukhov length, differences in the ABL
behavior were seen between the very stable boundary layer cases (5 m/s and 10
m/s), and the stably stratified case (15 m/s).  The very stable cases had
shorter turbulent structures and less growth in the boundary layer compared to
the 15 m/s case.  The behavior of the three stable boundary layers was also
compared to the neutral and unstable cases studied in \cite{cheung2020large}.
As expected, the turbulent structures in all stable cases were noticeably
smaller than in the neutral and unstable boundary layers.

In addition to these simulations, we also examined the domain and mesh
requirements necessary for the stable offshore boundary layers
considered here.  Based on the calculated turbulent integral
length scale, horizontal domain sizes of 750 m $\times$ 750 m were found
to be necessary for these simulations.  Various mesh resolutions were
also considered for the stable runs, and it was determined that the
finest resolution of $\Delta_x$=2.5 m was able to resolve the longitudinal
length scales and highest temporal frequencies, but inadequate for
resolving the turbulence in the lateral direction.  Further study is
warranted to comprehensively determine the impact of stable
stratification on the required mesh resolution.

Lastly, a comparison of the ExaWind solvers Nalu-Wind and AMR-Wind was
conducted using the stable offshore boundary layer at 5 m/s.  The mean
statistics were found to agree well between the two LES codes.  Small
differences were found in wind profiles at higher elevations and for
the peak $S_u$ spectra, and excellent matches were observed for the
averaged veer, temperature, and TI profiles.  For the problem sizes
considered here, it was also found that AMR-Wind was significantly
more computationally efficient than Nalu-Wind when measured on a time
per degree of freedom basis.  However, the good agreement is a
promising indication that the two codes can be used together
effectively for future wind farm applications, where the highly
efficient, structured nature of AMR-Wind handles the free stream
boundary layer and is combined with the unstructured Nalu-Wind solver
for complex turbine geometries.

% The performance difference may be caused
% by XXX, but warrants further investigation in future work.

%% \section*{Appendix}

%% An Appendix, if needed, should appear before the acknowledgments.

\section*{Acknowledgments}

The authors wish to acknowledge the contributions from M. Churchfield
and S. Yellapantula for their development of the LES boundary
conditions and for their efforts in validating the ExaWind codes.

This research was supported by the Wind Energy Technologies Office of the
U.S. Department of Energy (DOE) Office of Energy Efficiency and Renewable
Energy.  Sandia National Laboratories is a multimission laboratory managed and
operated by National Technology \& Engineering Solutions of Sandia, LLC, a
wholly owned subsidiary of Honeywell International Inc., for the U.S. Department
of Energy's National Nuclear Security Administration under contract
DE-NA0003525. The views expressed in the article do not necessarily represent
the views of the U.S. Department of Energy or the United States Government.

This work was authored in part by the National Renewable Energy Laboratory,
operated by Alliance for Sustainable Energy, LLC, for the U.S. Department of
Energy (DOE) under Contract No. DE-AC36- 08GO28308. M. Brazell was funded by the
Exascale Computing Project (17-SC-20-SC), a collaborative effort of two DOE
organizations (Office of Science and the National Nuclear Security
Administration).

A portion of the research was performed using computational resources sponsored
by the DOE's Office of Energy Efficiency and Renewable Energy and located at the
National Renewable Energy Laboratory.

\bibliography{references}

\end{document}
